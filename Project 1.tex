\documentclass[a4paper, 12pt]{article}
\usepackage[top=0.5in,bottom=0.5in,left=0.5in,right=0.5in]{geometry}
\usepackage[utf8]{inputenc}
\usepackage{amsmath}
\usepackage{graphicx}
\usepackage[font=small,labelfont=bf]{caption}
\usepackage{algorithm}
\usepackage[noend]{algpseudocode}

\makeatletter
\def\BState{\State\hskip-\ALG@thistlm}
\makeatother

\usepackage{listings}
\usepackage{color}

\definecolor{dkgreen}{rgb}{0,0.6,0}
\definecolor{gray}{rgb}{0.5,0.5,0.5}
\definecolor{mauve}{rgb}{0.58,0,0.82}

\lstset{frame=tb,
  language=C++,
  aboveskip=3mm,
  belowskip=3mm,
  showstringspaces=false,
  columns=flexible,
  basicstyle={\small\ttfamily},
  numbers=none,
  numberstyle=\tiny\color{gray},
  keywordstyle=\color{blue},
  commentstyle=\color{dkgreen},
  stringstyle=\color{mauve},
  breaklines=true,
  breakatwhitespace=true,
  tabsize=3
}

\title{CS 341 - Programming Project 1}
\author{Awnon K. Bhowmik}

\begin{document}
\maketitle
\noindent\rule{\textwidth}{1pt}
\textbf{The Battery Powered Car Problem}

\section*{The Setting}
\begin{itemize}
	\item There is one battery powered car
	\item There is a pile of $n$ batteries
	\item The maximum mileage of the car, is $k$ km per battery
	\item The car can carry a maximum of one spare battery
	\item The car starts from a point $S$ and goes back and forth to displace the batteries, and itself a certain distance of $x$ km from $S$
\end{itemize}

\section*{The Question}
	Find the maximum displacement from the starting point $S$ for a given pile of $n$ batteries.
	
\section*{Analysis}
\begin{flushleft}
Given $n$ batteries, if the car has a mileage of $k$ km on one battery, then on $n$ batteries, it will travel a maximum of $nk$ km. Moving in this manner means, with an odd number of batteries, the total displacement will be $k$ km, whereas with an even number of batteries, the displacement will essentially be $0$.
\end{flushleft}

\begin{flushleft}
This is why we are forced to take another way out. Suppose we move $x$ km with the pile of batteries. The car uses one, so the other $(n-1)$ batteries are moved a distance of $x$ km one at a time. The round trip is $2x$ km for the first $n-2$ batteries.
\end{flushleft}

\begin{flushleft}
The effective displacement from $S$ is given by the following algebraic equation $$2x(n-2)+x=k$$ where $x$ is the one way trip in km. Solving for $x$ gives...
\begin{align}
x&=\dfrac{k}{2n-3}
\end{align}
Equation (1) can be turned into one with a \textbf{recursive nature}, by simple algebraic manipulations.

Suppose that we rewrite equation (1) as a sequence
$$x_n=\dfrac{k}{2n-3}$$\\
The next term can then be calculated as....\\
\begin{align*}
x_{n+1}&=\dfrac{k}{2(n+1)-3}\\\dfrac{1}{x_{n+1}}&=\dfrac{2n-3+2}{k}\\\dfrac{1}{x_{n+1}}&=\dfrac{2n-3}{k}+\dfrac{2}{k}\\\dfrac{1}{x_{n+1}}&=\dfrac{1}{x_n}+\dfrac{2}{k}\\x_{n+1}&=\dfrac{1}{\dfrac{1}{x_n}+\dfrac{2}{k}}\qquad\textbf{This is what we'll use}\\x_{n+1}&=\dfrac{kx_n}{2x_n+k}\qquad\begin{cases}\textbf{Using this requires two recursive calls}\\\textbf{which slows down execution time by a lot}\end{cases}
\end{align*}
\end{flushleft}

\section*{The $C++$ implementation}
\textbf{Algorithm}
\begin{itemize}
\item We pre-define the mileage $k$ of the car running on a single battery. This is typically a 100 miles for an average battery powered car, with the exception of Tesla Model S that averages about 250 miles on a single charge.
\item We read in a battery pile size $n$.
\item While $n>0$ we do the following
\end{itemize}
\begin{center}
$x_{n+1}=\begin{cases}1&n=1,2\\\dfrac{kx_n}{2x_n+k}&\text{otherwise}\end{cases}$
\end{center}
\begin{itemize}
\item We make the program display the battery remaining and the distance traveled. This step is optional, we can only print out given the initial pile size $n$ and the distance traveled $x$.
\end{itemize}

\begin{algorithm*}
\caption{Pseudocode for algorithm}\label{euclid}
\begin{algorithmic}[1]
\Procedure{MyProcedure}{}
\State $\textit{k} \gets \text{distance traveled in single charge}$
\State $n \gets \text{battery pile size}$
\State $x \gets 0 \text{ Initially distance traveled in one direction is zero}$
\If {$n = 1$ or $n = 2$} \Return $k$
\Else {} \Return $\dfrac{kx_n}{2x_n+k}$
\EndIf
\BState \emph{loop}:
\State $x \gets x+\dfrac{kx_n}{2x_n+k}$
\State \textbf{print $n,x$}
\State \textbf{goto} \emph{loop}
\State \textbf{close}
\EndProcedure
\end{algorithmic}
\end{algorithm*}

\pagebreak
\textbf{The C++ Code}
\begin{lstlisting}
#include <iostream>
#include <iomanip>
constexpr auto k = 100;

using namespace std;

double dist(int n)
{
    if (n == 1 || n == 2)
        return k;
    else
        return (1.0 / (1.0 / dist(n - 1) + 2.0 / k));
}

int main()
{
    int n;
    double x = 0;
    cout << "Enter the number of batteries:";
    cin >> n;

    cout << "Batteries"
         << "\tDistance" << endl;
    while (n > 0)
    {
        x += dist(n);
        cout << setw(5) << n << setw(15) << x << endl;
        n--;
    }

    return 0;
}
\end{lstlisting}

\section*{Data Analysis}
\begin{flushleft}
	The following is a sample output data from the program.
	\begin{center}
	\includegraphics[scale=0.77]{"Car Battery".jpg}\captionof*{"Car Battery".jpg}{Table 1}
	\end{center}
	
	According to the data from Table 1, it seems like with more and more batteries, the car can initially move only a little, but it gradually covers more distance with the remaining batteries, and drastically when the batteries are about to run out. This is exactly what we were supposed to get, the situation model turned out as such.
	
	This also tells us that $$x\propto \dfrac{1}{n}$$ where $n$ represents the number of batteries remaining in the pile. The input of $20$ was chosen at random. Initially there were some thoughts of splitting up into even and odd inputs, but the way the code is written took care of it after all. Basically, the car initially travels shorter distance since it has to compensate going back and forth to move the pile of batteries. As a result, it \textit{loses effective distance traveled forth} by quite a lot. This can be found by just comparing the distance we have from our output with the theoretical distance possible. This is $$\textbf{difference} = 20\cdot 100 - 345.403 = 1654.597$$ and $$\textbf{percentage difference}=\dfrac{1654.597}{2000}\times 100 = 82.7\%$$
	Needless to say, this error increases (logarithmically) as the starting number of batteries in the pile increases. Let's look at a table...
	
	\begin{center}
	    \begin{tabular}{|c|c|c|c|}\hline
	     Batteries&	Theoretical Distance&	Actual Distance&	Percentage Difference\\
	     1	&100	&100	&0\\
2	&200	&200	&0\\
3	&300	&233.333	&22.22233333\\
4	&400	&253.333	&36.66675\\
5	&500	&267.619	&46.4762\\
6	&600	&278.73	&53.545\\
7	&700	&287.821	&58.88271429\\
8	&800	&295.513	&63.060875\\
9	&900	&302.18	&66.42444444\\
10	&1000	&308.062	&69.1938\\
11	&1100	&313.326	&71.51581818\\
12	&1200	&318.087	&73.49275\\
13	&1300	&322.435	&75.19730769\\
14	&1400	&326.435	&76.68321429\\
15	&1500	&330.139	&77.99073333\\
16	&1600	&333.587	&79.1508125\\
17	&1700	&336.813	&80.18747059\\
18	&1800	&339.843	&81.11983333\\
19	&1900	&342.701	&81.96310526\\
20	&2000	&345.403	&82.72985\\\hline
	    \end{tabular}
	\end{center}
	Here's a visual
	\begin{center}
	    \includegraphics[scale=0.9]{"Graph 3".jpg}
	\end{center}
	\begin{center}
	\includegraphics[scale=0.9]{"Graph 1".jpg}
	\end{center}
	
	Our original C++ code can be tweaked so that it gives us the maximum distance traveled for a number of batteries in the starting pile. This time we will use a custom defined input, and not user input to keep it consistent with our previous experimental results. The code is as follows....
\end{flushleft}

\textbf{A variation of the C++ code}
\begin{lstlisting}
#include <iostream>
#include <iomanip>
constexpr auto k = 100;

using namespace std;

double dist(int n)
{
	if (n == 1 || n == 2)
		return k;
	else
		return (1.0 / (1.0 / dist(n - 1) + 2.0 / k));
}

int main()
{
	int n;
	double x = 0;

	cout << "Batteries"
		<< "   Distance" << endl;
	
	/* This gives the maximum distance that can be
	traveled by the car with n number of batteries
	in the initial pile */
	for (int m = 1; m <= 20; m++) {
		n = m;
		while (n > 0)
		{
			x += dist(n);
			n--;
		}
		cout << setw(5) << m << setw(15) << x << endl;
		x = 0;
	}

	return 0;
}
\end{lstlisting}

\pagebreak
\begin{flushleft}
Here is a snapshot of the output
\begin{center}
	\includegraphics[scale=1]{"Car Battery 2".jpg}\captionof*{"Car Battery 2".jpg}{Table 2}
\end{center}
A graph plotted from the results in Table 2 is given below
\begin{center}
	\includegraphics[scale=1]{"Graph 2".jpg}
\end{center}

It seems like a logarithmic trend line is close to the best fit for this graph. This is what was expected just by looking at the output data.
\end{flushleft}

\end{document}